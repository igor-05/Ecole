\documentclass[a4paper, 12pt]{article}
\usepackage{a4}
\usepackage[utf8]{inputenc}
\usepackage[french]{babel}
\usepackage[T1]{fontenc}
\usepackage{graphicx}
\usepackage{amsmath}
\usepackage{amssymb}
\usepackage{float}

\DeclareUnicodeCharacter{2212}{-}
\author{Igor De Bock}
\title{Elizabeth b\'athory}
\parskip=10pt
\begin{document}
    \maketitle
    %intro
    Elizabeth Bathory est un comtesse hongroise de transylvanie née le 7
    août 1560 et morte le 21 août 1614. Aujourd'hui, elle plus communément
    connue sous le nom de la \textit{comptesse sanglangte} et fait encore parler
    d'elle pour plusieurs raisons.
    
    %plus meurtrière et sadique
    Premièrement, car elle est une des plus légendaires et sadiques meutrières
    de l'histoire. En effet, elle aurait torturé à mort des centaines de jeunes
    filles au cours sa vie. Le témoignage le plus élevé, celui de sa servante,
    indiquant un nombre de 650 victimes. Les témoignages de ses manières de
    procéder, toutes plus cruelles et imaginatives les unes des autres,
    dépassaient tout ce qu'on avait pu voir et auraient fait vomir les
    spectateurs du procès, car oui, après des années passée en impunité elle fut
    jugée. Elle aurait, par exemple, forcé une de ses victimes à se couper un
    membre, le cuire puis le manger. Mordu à mort une de ses servantes pour avoir
    volé une poire. Et pour les "crimes" qu'elle jugeait moins graves, elle
    chaufferait une pièce à blanc et forcerai la servante à la tenir dans sa
    paume fermée. Torturer ses servantes était l'un de ses passes temps quand
    son mari n'était pas là, ce qui était partagé par ses proches (nourrisse,
    confidente).

    %Femme puissante
    Elisabeth est issue d'une grande famille noble de Hongrie, les Bathory. Sa
    famille controlait la transilvanie, une énorme région de l'europe centrale
    faisant près de 3 fois la taille de la Belgique. En outre, son oncle était 
    roi de Pologne et elle, une des plus fidèles amie de l'empereur.

    Elle se marie à un grand seigneur de hongrie, le comte \textit{Nadasdy} 
    appartenant à une famille qui partage le pouvoir locale. 
    
    Quand celui-ci meurt, elle hérite de son chateau et de toute sa fortune
    devenant ainsi l'une des personnes les puissantses et riches du pays. Plus
    personne n'ose s'opposé à elle.

    %affaire mystère
    Comme discuté dans le point précédent, elle était une femme puissante, et
    intelligente en possesion d'un territoire convoité. Son jugement eu des
    repercussions politiques majeurs.

    Il n'y a peu de doutes, Elizabeth était sadique et a surement abusé de ses 
    servantes, mais la différence entre la vérité et la légende n'est pas si 
    simple.

    Pour certains historiens, elle la femme la plus meurtrière de l'histoire
    et la personne la plus meurtrière de l'occident. Et pour d'autres, elle est
    innocente. Le roi lui a tendu un piege car si elle mourrait il pouvait
    s'emparer de son territoire.

    De plus, elle n'a pas pu testifié à son procès, de nombreux témoignages ont
    étés obtenus sous la torture et son frauduleux, et de nombreux documents ont
    été perdus. On ne connait par exemple pas un seul nom de victime.

    Même si les légendes persistes, il est impossible de determiner à sans
    pourcent sa culpabilitée, ce qui entretien l'intérêt des historiens pour
    elle.

    %inspiré culture puis livres.
    Au 16ème siècle, La Hongrie était plongée dans un monde de croyances. Le
    peuple était persuadé de l'existence des sorcières, des fées et des 
    esprits malins. Des comptes et rumeurs horrifiques les décrivaient.
    
    C'est dans ce contexte, qu'après sa mort, elle nourrisa les légendes
    locales, et particulièrement celle des vampires.

    Elisabeth à toujours était considérée comme une des plus belles femmes du
    royaume. Un jour, elle vit une vieille femme laide et se moqua d'elle. La
    femme lui réponda que dans peu de temps elle sera comme elle. C'est là que
    débuta l'obsession d'Elisabeth pour la beautée et la jeunesse.

    Elle commença par essayer différents types de baumes, crèmes et autres
    produits vendu comme rajeunissant. Mais ceux-ci ne semblaient pas
    fonctionner.

    Un jours, elle renda visite à une ancienne amie, une femme considérée comme
    une sorcière. Celle-ci lui dira que le meilleur remède contre la vieillesse
    est le sang de vierge.

    A partir de ce moment Bathory, ayant déjà des tendances sadiques commença à
    attirer des jeunes filles dans son chateau, les tuer et récolter leur sang.

    Certains témoignages disent même qu'elle mordit, buvait et se baignait dans
    le sang de ses victimes. Sa femme de ménage indiquant qu'il y avait telement
    de sang dans le chateau, que la seule manière de le nétoyer était de jeter
    de la cendre par terre pour l'absorber.

    Après sa mort, de nombreuses personnes disait l'avoir vu roder la nuit.
    C'est là qu'est né le mythe du vampire, vivant éternelement du sang de
    victimes et de la transilvanie.

    Le livre Dracula va particulièrement s'inspiré d'elle. On peut y voir de
    nombreuses ressemblances mis à part leur sexes opposés. Ils sont tout deux 
    des comptes vivant seuls dans un chateau perché au sommet d'une colline de
    transilvanie.

    Dracula, elisabeth, les histoires de vampires et la transilvanie vont 
    passioné les auteurs de monuments de l'horreur, et même encore récemment,
    des films comme "la comtesse" de 2009 s'inspire de son histoire.

    Si elle n'a pas laissée énormément de traces directes dans notre culture,
    elle l'a massivement influencée, et est la source de nombreuses histoire
    connues de tout le monde.

    %illustre dicton
    Elle fut fiancée à l'age de 11 ans et vit (supposément) son amant, de plus
    basse classe, avec lequle elle eut une grosse illégitime, se faire castrer
    et dévorer par des chiens.

    Son belle-oncle était quelqu'un de très sadique tout comme son mari qui lui 
    apprena jeune des techniques de tortures. Ses deux oncles étaient fous et
    sa famille était remplie de relation incestueuses.

    Sa tante tua deux de ses mari et finira torturée et violée collectivement en
    Turqui.

    Enfant, elle vit des scènes de tortures courantes dans son village et
    n'était jamais punie pour son comportement.

    Tout ça pour dire qu'elle n'a pas jouit de la meilleur éducation et
    condition de vie pour se dévellopper. Ainsi, très jeune elle dévellopa un
    penchant pour la torture, particulièrement pour ses servantes, ce qui
    semblait même encouragé par
    sa famille.

    %conclusion

    En conclusion, Elisabeth est un 
\end{document}