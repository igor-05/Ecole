\documentclass[a4paper, 12pt]{article}
\usepackage{a4}
\usepackage[utf8]{inputenc}
\usepackage[french]{babel}
\usepackage[T1]{fontenc}
\usepackage{graphicx}
\usepackage{float}
\usepackage{chemfig}
\renewcommand{\arraystretch}{1.5}
\usepackage[version=4]{mhchem} % Package for chemical equation typesetting
\usepackage{amssymb}
\usepackage{siunitx}

\author{Igor et Antoine}
\date{\today}
\title{Compte rendu : Traveaux pratiques thermochimie} 
\parskip=5pt
\begin{document}
\maketitle
\tableofcontents
\section{Vérification de la loi de Hess}
\subsection{Mesures}
Les mesures suivantes sont prises du groupe Camille/Dune car notre équipe n'en
a pas.
\subsubsection{Première réaction}
Rappel de la réaction : 
\begin{center}
    \ce{NaOH_{(sol)} ->[\ce{+H_2O}] Na+ + OH-}
\end{center}

\begin{itemize}
    \item[Masse \ce{H_2O} : ] 0.1003 kg
    \item[Masse NaOH : ] 0.00212 kg
    \item[Température initiale \ce{H_2O} ($Ti$) :] 23°C  
    \item[Température finale ($Tf$) :] 27°C  
    \item[Différence de température ($\Delta T$) :] 4°C  
\end{itemize}

\subsubsection{Deuxième réaction}
Rappel de la réaction : 
\begin{center}
    \ce{NaOH_{(sol)} + H+ + Cl- -> Na+ + Cl- + H_2O} 
\end{center}

\begin{itemize}
    \item[Masse \ce{H_2O} : ] 0.10377 kg
    \item[Masse NaOH : ] 0.00202 kg
    \item[Température initiale \ce{HCl} ($Ti$) :] 23°C 
    \item[Température finale ($Tf$) :] 32°C  
    \item[Différence de température ($\Delta T$) :] 9°C  
\end{itemize}

\subsubsection{Troisième réaction}
Rappel de la réaction : 
\begin{center}
    \ce{Na+ + OH- + H+ + Cl- -> Na+ + Cl- + H_2O} 
\end{center}

\begin{itemize}
    \item[Masse HCL : ] 0.10038 kg
    \item[Température initiale \ce{HCl} ($Ti$) :] 23°C 
    \item[Température finale ($Tf$) :] 29°C  
    \item[Différence de température ($\Delta T$) :] 6°C  
\end{itemize}

\subsection{Calculs}
\subsubsection{Enthaplie première réaction}
\begin{align*}
    \Delta H1 \approx 4180 Jkg^{-1}K^{-1} \cdot 0.1003kg \cdot 4K \approx 1677 J
\end{align*}

\subsubsection{Enthalpie deuxième réaction}
\begin{align*}
    \Delta H2 \approx 4180 Jkg^{-1}K^{-1} \cdot 0.1038kg \cdot 9K \approx 3905 J
\end{align*}

\subsubsection{Enthalpie troisième réaction}
\begin{align*}
    \Delta H3 \approx 4180 Jkg^{-1}K^{-1} \cdot 0.1004kg \cdot 6K \approx 2518 J
\end{align*}

\subsection{Résultats}
Selon la loi de \textit{Hess}, l'enthalpie est une fonction d'état, sa variation
ne dépend que de la différence entre l'état initial et final.

Ainsi, 

\section{Déterminer la température de combustion du magnésium}

\end{document}
